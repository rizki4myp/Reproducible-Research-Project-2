% Options for packages loaded elsewhere
\PassOptionsToPackage{unicode}{hyperref}
\PassOptionsToPackage{hyphens}{url}
%
\documentclass[
]{article}
\usepackage{amsmath,amssymb}
\usepackage{lmodern}
\usepackage{ifxetex,ifluatex}
\ifnum 0\ifxetex 1\fi\ifluatex 1\fi=0 % if pdftex
  \usepackage[T1]{fontenc}
  \usepackage[utf8]{inputenc}
  \usepackage{textcomp} % provide euro and other symbols
\else % if luatex or xetex
  \usepackage{unicode-math}
  \defaultfontfeatures{Scale=MatchLowercase}
  \defaultfontfeatures[\rmfamily]{Ligatures=TeX,Scale=1}
\fi
% Use upquote if available, for straight quotes in verbatim environments
\IfFileExists{upquote.sty}{\usepackage{upquote}}{}
\IfFileExists{microtype.sty}{% use microtype if available
  \usepackage[]{microtype}
  \UseMicrotypeSet[protrusion]{basicmath} % disable protrusion for tt fonts
}{}
\makeatletter
\@ifundefined{KOMAClassName}{% if non-KOMA class
  \IfFileExists{parskip.sty}{%
    \usepackage{parskip}
  }{% else
    \setlength{\parindent}{0pt}
    \setlength{\parskip}{6pt plus 2pt minus 1pt}}
}{% if KOMA class
  \KOMAoptions{parskip=half}}
\makeatother
\usepackage{xcolor}
\IfFileExists{xurl.sty}{\usepackage{xurl}}{} % add URL line breaks if available
\IfFileExists{bookmark.sty}{\usepackage{bookmark}}{\usepackage{hyperref}}
\hypersetup{
  hidelinks,
  pdfcreator={LaTeX via pandoc}}
\urlstyle{same} % disable monospaced font for URLs
\usepackage[margin=1in]{geometry}
\usepackage{color}
\usepackage{fancyvrb}
\newcommand{\VerbBar}{|}
\newcommand{\VERB}{\Verb[commandchars=\\\{\}]}
\DefineVerbatimEnvironment{Highlighting}{Verbatim}{commandchars=\\\{\}}
% Add ',fontsize=\small' for more characters per line
\usepackage{framed}
\definecolor{shadecolor}{RGB}{248,248,248}
\newenvironment{Shaded}{\begin{snugshade}}{\end{snugshade}}
\newcommand{\AlertTok}[1]{\textcolor[rgb]{0.94,0.16,0.16}{#1}}
\newcommand{\AnnotationTok}[1]{\textcolor[rgb]{0.56,0.35,0.01}{\textbf{\textit{#1}}}}
\newcommand{\AttributeTok}[1]{\textcolor[rgb]{0.77,0.63,0.00}{#1}}
\newcommand{\BaseNTok}[1]{\textcolor[rgb]{0.00,0.00,0.81}{#1}}
\newcommand{\BuiltInTok}[1]{#1}
\newcommand{\CharTok}[1]{\textcolor[rgb]{0.31,0.60,0.02}{#1}}
\newcommand{\CommentTok}[1]{\textcolor[rgb]{0.56,0.35,0.01}{\textit{#1}}}
\newcommand{\CommentVarTok}[1]{\textcolor[rgb]{0.56,0.35,0.01}{\textbf{\textit{#1}}}}
\newcommand{\ConstantTok}[1]{\textcolor[rgb]{0.00,0.00,0.00}{#1}}
\newcommand{\ControlFlowTok}[1]{\textcolor[rgb]{0.13,0.29,0.53}{\textbf{#1}}}
\newcommand{\DataTypeTok}[1]{\textcolor[rgb]{0.13,0.29,0.53}{#1}}
\newcommand{\DecValTok}[1]{\textcolor[rgb]{0.00,0.00,0.81}{#1}}
\newcommand{\DocumentationTok}[1]{\textcolor[rgb]{0.56,0.35,0.01}{\textbf{\textit{#1}}}}
\newcommand{\ErrorTok}[1]{\textcolor[rgb]{0.64,0.00,0.00}{\textbf{#1}}}
\newcommand{\ExtensionTok}[1]{#1}
\newcommand{\FloatTok}[1]{\textcolor[rgb]{0.00,0.00,0.81}{#1}}
\newcommand{\FunctionTok}[1]{\textcolor[rgb]{0.00,0.00,0.00}{#1}}
\newcommand{\ImportTok}[1]{#1}
\newcommand{\InformationTok}[1]{\textcolor[rgb]{0.56,0.35,0.01}{\textbf{\textit{#1}}}}
\newcommand{\KeywordTok}[1]{\textcolor[rgb]{0.13,0.29,0.53}{\textbf{#1}}}
\newcommand{\NormalTok}[1]{#1}
\newcommand{\OperatorTok}[1]{\textcolor[rgb]{0.81,0.36,0.00}{\textbf{#1}}}
\newcommand{\OtherTok}[1]{\textcolor[rgb]{0.56,0.35,0.01}{#1}}
\newcommand{\PreprocessorTok}[1]{\textcolor[rgb]{0.56,0.35,0.01}{\textit{#1}}}
\newcommand{\RegionMarkerTok}[1]{#1}
\newcommand{\SpecialCharTok}[1]{\textcolor[rgb]{0.00,0.00,0.00}{#1}}
\newcommand{\SpecialStringTok}[1]{\textcolor[rgb]{0.31,0.60,0.02}{#1}}
\newcommand{\StringTok}[1]{\textcolor[rgb]{0.31,0.60,0.02}{#1}}
\newcommand{\VariableTok}[1]{\textcolor[rgb]{0.00,0.00,0.00}{#1}}
\newcommand{\VerbatimStringTok}[1]{\textcolor[rgb]{0.31,0.60,0.02}{#1}}
\newcommand{\WarningTok}[1]{\textcolor[rgb]{0.56,0.35,0.01}{\textbf{\textit{#1}}}}
\usepackage{longtable,booktabs,array}
\usepackage{calc} % for calculating minipage widths
% Correct order of tables after \paragraph or \subparagraph
\usepackage{etoolbox}
\makeatletter
\patchcmd\longtable{\par}{\if@noskipsec\mbox{}\fi\par}{}{}
\makeatother
% Allow footnotes in longtable head/foot
\IfFileExists{footnotehyper.sty}{\usepackage{footnotehyper}}{\usepackage{footnote}}
\makesavenoteenv{longtable}
\usepackage{graphicx}
\makeatletter
\def\maxwidth{\ifdim\Gin@nat@width>\linewidth\linewidth\else\Gin@nat@width\fi}
\def\maxheight{\ifdim\Gin@nat@height>\textheight\textheight\else\Gin@nat@height\fi}
\makeatother
% Scale images if necessary, so that they will not overflow the page
% margins by default, and it is still possible to overwrite the defaults
% using explicit options in \includegraphics[width, height, ...]{}
\setkeys{Gin}{width=\maxwidth,height=\maxheight,keepaspectratio}
% Set default figure placement to htbp
\makeatletter
\def\fps@figure{htbp}
\makeatother
\setlength{\emergencystretch}{3em} % prevent overfull lines
\providecommand{\tightlist}{%
  \setlength{\itemsep}{0pt}\setlength{\parskip}{0pt}}
\setcounter{secnumdepth}{-\maxdimen} % remove section numbering
\ifluatex
  \usepackage{selnolig}  % disable illegal ligatures
\fi

\author{}
\date{\vspace{-2.5em}}

\begin{document}

\#Reproducible Research Course Project 2

\#\#Analysis of the U.S. National Oceanic and Atmospheric
Administration?s (NOAA) storm database

\#\#\#\#This project explores the NOAA storm database, which tracks
major storms and weather events, to address the most severe types of
weather events in the USA, which caused greatest damage to human
population in terms of fatalities/injuries and economic loss during the
years 1950 - 2011.

\#\#\#\#There are two goals of this analysis:

\begin{itemize}
\tightlist
\item
  identify the weather events that are most harmful with respect to
  population health
\item
  identify the weather events that have the greatest economic
  consequences.
\end{itemize}

\#\#\#\#Based on our analysis, we conclude that TORNADOS and FLOODS are
most harmful weather events in the USA in terms of the risk to human
health and economic impact.

\#\#Data Processing

\#\#\#\#The data source is in the form of a comma-separated-value file
compressed via the bzip2 algorithm to reduce its size. It is possible to
download the source file from the course web site:
\href{https://d396qusza40orc.cloudfront.net/repdata\%2Fdata\%2FStormData.csv.bz2}{Storm
Data}

\begin{Shaded}
\begin{Highlighting}[]
\CommentTok{\# downloading data}
\NormalTok{Url\_data }\OtherTok{\textless{}{-}} \StringTok{"https://d396qusza40orc.cloudfront.net/repdata\%2Fdata\%2FStormData.csv.bz2"}

\NormalTok{File\_data }\OtherTok{\textless{}{-}} \StringTok{"StormData.csv.bz2"}
\ControlFlowTok{if}\NormalTok{ (}\SpecialCharTok{!}\FunctionTok{file.exists}\NormalTok{(File\_data)) \{}
        \FunctionTok{download.file}\NormalTok{(Url\_data, File\_data, }\AttributeTok{mode =} \StringTok{"wb"}\NormalTok{)}
\NormalTok{\}}

\CommentTok{\# reading data}
\NormalTok{Raw\_data }\OtherTok{\textless{}{-}} \FunctionTok{read.csv}\NormalTok{(}\AttributeTok{file =}\NormalTok{ File\_data, }\AttributeTok{header=}\ConstantTok{TRUE}\NormalTok{, }\AttributeTok{sep=}\StringTok{","}\NormalTok{)}
\end{Highlighting}
\end{Shaded}

\#\#\#\#Additional documentation on the database was provided here:

\begin{itemize}
\tightlist
\item
  \href{https://d396qusza40orc.cloudfront.net/repdata\%2Fpeer2_doc\%2Fpd01016005curr.pdf}{National
  Weather Service Storm Data Documentation}
\item
  \href{https://d396qusza40orc.cloudfront.net/repdata\%2Fpeer2_doc\%2FNCDC\%20Storm\%20Events-FAQ\%20Page.pdf}{National
  Climatic Data Center Storm Events FAQ}
\item
  \href{https://www.coursera.org/learn/reproducible-research/discussions/weeks/4/threads/IdtP_JHzEeaePQ71AQUtYw}{Mentor?s
  comments in the Discussion Forum on the Course web-site}
\end{itemize}

\#\#\#\#1. According to NOAA, the data recording start from Jan.~1950.
At that time, they recorded only one event type - tornado. They added
more events gradually, and only from Jan 1996 they started recording all
events type. Since our objective is comparing the effects of different
weather events, we need only to include events that started not earlier
than Jan 1996.

\begin{Shaded}
\begin{Highlighting}[]
\CommentTok{\# subsetting by date}
\NormalTok{Main\_data }\OtherTok{\textless{}{-}}\NormalTok{ Raw\_data}
\NormalTok{Main\_data}\SpecialCharTok{$}\NormalTok{BGN\_DATE }\OtherTok{\textless{}{-}} \FunctionTok{strptime}\NormalTok{(Raw\_data}\SpecialCharTok{$}\NormalTok{BGN\_DATE, }\StringTok{"\%m/\%d/\%Y \%H:\%M:\%S"}\NormalTok{)}
\NormalTok{Main\_data }\OtherTok{\textless{}{-}} \FunctionTok{subset}\NormalTok{(Main\_data, BGN\_DATE }\SpecialCharTok{\textgreater{}} \StringTok{"1995{-}12{-}31"}\NormalTok{)}
\end{Highlighting}
\end{Shaded}

\#\#\#\#2. Based on the above mentioned documentation and preliminary
exploration of raw data with ?str?, ?names?, ?table?, ?dim?, ?head?,
?range? and other similar functions we can conclude that there are 7
variables we are interested in regarding the two questions.

\#\#\#\#Namely: EVTYPE, FATALITIES, INJURIES, PROPDMG, PROPDMGEXP,
CROPDMG, CROPDMGEXP.

\#\#\#\#Therefore, we can limit our data to these variables.

\begin{Shaded}
\begin{Highlighting}[]
\NormalTok{Main\_data }\OtherTok{\textless{}{-}} \FunctionTok{subset}\NormalTok{(Main\_data, }\AttributeTok{select =} \FunctionTok{c}\NormalTok{(EVTYPE, FATALITIES, INJURIES, PROPDMG, PROPDMGEXP, CROPDMG, CROPDMGEXP))}
\end{Highlighting}
\end{Shaded}

\#\#\#\#Contents of data now are as follows:

EVTYPE ? type of event\\
FATALITIES ? number of fatalities\\
INJURIES ? number of injuries\\
PROPDMG ? the size of property damage\\
PROPDMGEXP - the exponent values for `PROPDMG' (property damage)\\
CROPDMG - the size of crop damage\\
CROPDMGEXP - the exponent values for `CROPDMG' (crop damage)

\#\#\#\#3. There are almost 1000 unique event types in EVTYPE column.
Therefore, it is better to limit database to a reasonable number. We can
make it by capitalizing all letters in EVTYPE column as well as
subsetting only non-zero data regarding our target numbers.

\begin{Shaded}
\begin{Highlighting}[]
\CommentTok{\#cleaning event types names}
\NormalTok{Main\_data}\SpecialCharTok{$}\NormalTok{EVTYPE }\OtherTok{\textless{}{-}} \FunctionTok{toupper}\NormalTok{(Main\_data}\SpecialCharTok{$}\NormalTok{EVTYPE)}

\CommentTok{\# eliminating zero data}
\NormalTok{Main\_data }\OtherTok{\textless{}{-}}\NormalTok{ Main\_data[Main\_data}\SpecialCharTok{$}\NormalTok{FATALITIES }\SpecialCharTok{!=}\DecValTok{0} \SpecialCharTok{|} 
\NormalTok{                       Main\_data}\SpecialCharTok{$}\NormalTok{INJURIES }\SpecialCharTok{!=}\DecValTok{0} \SpecialCharTok{|} 
\NormalTok{                       Main\_data}\SpecialCharTok{$}\NormalTok{PROPDMG }\SpecialCharTok{!=}\DecValTok{0} \SpecialCharTok{|} 
\NormalTok{                       Main\_data}\SpecialCharTok{$}\NormalTok{CROPDMG }\SpecialCharTok{!=}\DecValTok{0}\NormalTok{, ]}
\end{Highlighting}
\end{Shaded}

\#\#\#\#Now we have 186 unique event types and it seems like something
to work with.

\#\#Population health data processing

\#\#\#\#We aggregate fatalities and injuries numbers in order to
identify TOP-10 events contributing the total people loss:

\begin{Shaded}
\begin{Highlighting}[]
\NormalTok{Health\_data }\OtherTok{\textless{}{-}} \FunctionTok{aggregate}\NormalTok{(}\FunctionTok{cbind}\NormalTok{(FATALITIES, INJURIES) }\SpecialCharTok{\textasciitilde{}}\NormalTok{ EVTYPE, }\AttributeTok{data =}\NormalTok{ Main\_data, }\AttributeTok{FUN=}\NormalTok{sum)}
\NormalTok{Health\_data}\SpecialCharTok{$}\NormalTok{PEOPLE\_LOSS }\OtherTok{\textless{}{-}}\NormalTok{ Health\_data}\SpecialCharTok{$}\NormalTok{FATALITIES }\SpecialCharTok{+}\NormalTok{ Health\_data}\SpecialCharTok{$}\NormalTok{INJURIES}
\NormalTok{Health\_data }\OtherTok{\textless{}{-}}\NormalTok{ Health\_data[}\FunctionTok{order}\NormalTok{(Health\_data}\SpecialCharTok{$}\NormalTok{PEOPLE\_LOSS, }\AttributeTok{decreasing =} \ConstantTok{TRUE}\NormalTok{), ]}
\NormalTok{Top10\_events\_people }\OtherTok{\textless{}{-}}\NormalTok{ Health\_data[}\DecValTok{1}\SpecialCharTok{:}\DecValTok{10}\NormalTok{,]}
\NormalTok{knitr}\SpecialCharTok{::}\FunctionTok{kable}\NormalTok{(Top10\_events\_people, }\AttributeTok{format =} \StringTok{"markdown"}\NormalTok{)}
\end{Highlighting}
\end{Shaded}

\begin{longtable}[]{@{}llrrr@{}}
\toprule
& EVTYPE & FATALITIES & INJURIES & PEOPLE\_LOSS \\
\midrule
\endhead
149 & TORNADO & 1511 & 20667 & 22178 \\
39 & EXCESSIVE HEAT & 1797 & 6391 & 8188 \\
48 & FLOOD & 414 & 6758 & 7172 \\
107 & LIGHTNING & 651 & 4141 & 4792 \\
153 & TSTM WIND & 241 & 3629 & 3870 \\
46 & FLASH FLOOD & 887 & 1674 & 2561 \\
146 & THUNDERSTORM WIND & 130 & 1400 & 1530 \\
182 & WINTER STORM & 191 & 1292 & 1483 \\
69 & HEAT & 237 & 1222 & 1459 \\
88 & HURRICANE/TYPHOON & 64 & 1275 & 1339 \\
\bottomrule
\end{longtable}

\#\#Economic consequences data processing

\#\#\#\#The number/letter in the exponent value columns (PROPDMGEXP and
CROPDMGEXP) represents the power of ten (10\^{}The number). It means
that the total size of damage is the product of PROPDMG and CROPDMG and
figure 10 in the power corresponding to exponent value.

\#\#\#\#Exponent values are:\\
- numbers from one to ten\\
- letters (B or b = Billion, M or m = Million, K or k = Thousand, H or h
= Hundred)\\
- and symbols ``-'', ``+'' and ``?'' which refers to less than, greater
than and low certainty. We have the option to ignore these three symbols
altogether.

\#\#\#\#We transform letters and symbols to numbers:

\begin{Shaded}
\begin{Highlighting}[]
\NormalTok{Main\_data}\SpecialCharTok{$}\NormalTok{PROPDMGEXP }\OtherTok{\textless{}{-}} \FunctionTok{gsub}\NormalTok{(}\StringTok{"[Hh]"}\NormalTok{, }\StringTok{"2"}\NormalTok{, Main\_data}\SpecialCharTok{$}\NormalTok{PROPDMGEXP)}
\NormalTok{Main\_data}\SpecialCharTok{$}\NormalTok{PROPDMGEXP }\OtherTok{\textless{}{-}} \FunctionTok{gsub}\NormalTok{(}\StringTok{"[Kk]"}\NormalTok{, }\StringTok{"3"}\NormalTok{, Main\_data}\SpecialCharTok{$}\NormalTok{PROPDMGEXP)}
\NormalTok{Main\_data}\SpecialCharTok{$}\NormalTok{PROPDMGEXP }\OtherTok{\textless{}{-}} \FunctionTok{gsub}\NormalTok{(}\StringTok{"[Mm]"}\NormalTok{, }\StringTok{"6"}\NormalTok{, Main\_data}\SpecialCharTok{$}\NormalTok{PROPDMGEXP)}
\NormalTok{Main\_data}\SpecialCharTok{$}\NormalTok{PROPDMGEXP }\OtherTok{\textless{}{-}} \FunctionTok{gsub}\NormalTok{(}\StringTok{"[Bb]"}\NormalTok{, }\StringTok{"9"}\NormalTok{, Main\_data}\SpecialCharTok{$}\NormalTok{PROPDMGEXP)}
\NormalTok{Main\_data}\SpecialCharTok{$}\NormalTok{PROPDMGEXP }\OtherTok{\textless{}{-}} \FunctionTok{gsub}\NormalTok{(}\StringTok{"}\SpecialCharTok{\textbackslash{}\textbackslash{}}\StringTok{+"}\NormalTok{, }\StringTok{"1"}\NormalTok{, Main\_data}\SpecialCharTok{$}\NormalTok{PROPDMGEXP)}
\NormalTok{Main\_data}\SpecialCharTok{$}\NormalTok{PROPDMGEXP }\OtherTok{\textless{}{-}} \FunctionTok{gsub}\NormalTok{(}\StringTok{"}\SpecialCharTok{\textbackslash{}\textbackslash{}}\StringTok{?|}\SpecialCharTok{\textbackslash{}\textbackslash{}}\StringTok{{-}|}\SpecialCharTok{\textbackslash{}\textbackslash{}}\StringTok{ "}\NormalTok{, }\StringTok{"0"}\NormalTok{,  Main\_data}\SpecialCharTok{$}\NormalTok{PROPDMGEXP)}
\NormalTok{Main\_data}\SpecialCharTok{$}\NormalTok{PROPDMGEXP }\OtherTok{\textless{}{-}} \FunctionTok{as.numeric}\NormalTok{(Main\_data}\SpecialCharTok{$}\NormalTok{PROPDMGEXP)}

\NormalTok{Main\_data}\SpecialCharTok{$}\NormalTok{CROPDMGEXP }\OtherTok{\textless{}{-}} \FunctionTok{gsub}\NormalTok{(}\StringTok{"[Hh]"}\NormalTok{, }\StringTok{"2"}\NormalTok{, Main\_data}\SpecialCharTok{$}\NormalTok{CROPDMGEXP)}
\NormalTok{Main\_data}\SpecialCharTok{$}\NormalTok{CROPDMGEXP }\OtherTok{\textless{}{-}} \FunctionTok{gsub}\NormalTok{(}\StringTok{"[Kk]"}\NormalTok{, }\StringTok{"3"}\NormalTok{, Main\_data}\SpecialCharTok{$}\NormalTok{CROPDMGEXP)}
\NormalTok{Main\_data}\SpecialCharTok{$}\NormalTok{CROPDMGEXP }\OtherTok{\textless{}{-}} \FunctionTok{gsub}\NormalTok{(}\StringTok{"[Mm]"}\NormalTok{, }\StringTok{"6"}\NormalTok{, Main\_data}\SpecialCharTok{$}\NormalTok{CROPDMGEXP)}
\NormalTok{Main\_data}\SpecialCharTok{$}\NormalTok{CROPDMGEXP }\OtherTok{\textless{}{-}} \FunctionTok{gsub}\NormalTok{(}\StringTok{"[Bb]"}\NormalTok{, }\StringTok{"9"}\NormalTok{, Main\_data}\SpecialCharTok{$}\NormalTok{CROPDMGEXP)}
\NormalTok{Main\_data}\SpecialCharTok{$}\NormalTok{CROPDMGEXP }\OtherTok{\textless{}{-}} \FunctionTok{gsub}\NormalTok{(}\StringTok{"}\SpecialCharTok{\textbackslash{}\textbackslash{}}\StringTok{+"}\NormalTok{, }\StringTok{"1"}\NormalTok{, Main\_data}\SpecialCharTok{$}\NormalTok{CROPDMGEXP)}
\NormalTok{Main\_data}\SpecialCharTok{$}\NormalTok{CROPDMGEXP }\OtherTok{\textless{}{-}} \FunctionTok{gsub}\NormalTok{(}\StringTok{"}\SpecialCharTok{\textbackslash{}\textbackslash{}}\StringTok{{-}|}\SpecialCharTok{\textbackslash{}\textbackslash{}}\StringTok{?|}\SpecialCharTok{\textbackslash{}\textbackslash{}}\StringTok{ "}\NormalTok{, }\StringTok{"0"}\NormalTok{, Main\_data}\SpecialCharTok{$}\NormalTok{CROPDMGEXP)}
\NormalTok{Main\_data}\SpecialCharTok{$}\NormalTok{CROPDMGEXP }\OtherTok{\textless{}{-}} \FunctionTok{as.numeric}\NormalTok{(Main\_data}\SpecialCharTok{$}\NormalTok{CROPDMGEXP)}

\NormalTok{Main\_data}\SpecialCharTok{$}\NormalTok{PROPDMGEXP[}\FunctionTok{is.na}\NormalTok{(Main\_data}\SpecialCharTok{$}\NormalTok{PROPDMGEXP)] }\OtherTok{\textless{}{-}} \DecValTok{0}
\NormalTok{Main\_data}\SpecialCharTok{$}\NormalTok{CROPDMGEXP[}\FunctionTok{is.na}\NormalTok{(Main\_data}\SpecialCharTok{$}\NormalTok{CROPDMGEXP)] }\OtherTok{\textless{}{-}} \DecValTok{0}
\end{Highlighting}
\end{Shaded}

\#\#\#\#At last, we create new values of total property damage and total
crop damage for analysis (we need ?dplr? package for that).

\begin{Shaded}
\begin{Highlighting}[]
\CommentTok{\#creating total damage values}
\FunctionTok{library}\NormalTok{(dplyr)}
\NormalTok{Main\_data }\OtherTok{\textless{}{-}} \FunctionTok{mutate}\NormalTok{(Main\_data, }
                    \AttributeTok{PROPDMGTOTAL =}\NormalTok{ PROPDMG }\SpecialCharTok{*}\NormalTok{ (}\DecValTok{10} \SpecialCharTok{\^{}}\NormalTok{ PROPDMGEXP), }
                    \AttributeTok{CROPDMGTOTAL =}\NormalTok{ CROPDMG }\SpecialCharTok{*}\NormalTok{ (}\DecValTok{10} \SpecialCharTok{\^{}}\NormalTok{ CROPDMGEXP))}
\end{Highlighting}
\end{Shaded}

\#\#\#\#Now we aggregate property and crop damage numbers in order to
identify TOP-10 events contributing the total economic loss:

\begin{Shaded}
\begin{Highlighting}[]
\NormalTok{Economic\_data }\OtherTok{\textless{}{-}} \FunctionTok{aggregate}\NormalTok{(}\FunctionTok{cbind}\NormalTok{(PROPDMGTOTAL, CROPDMGTOTAL) }\SpecialCharTok{\textasciitilde{}}\NormalTok{ EVTYPE, }\AttributeTok{data =}\NormalTok{ Main\_data, }\AttributeTok{FUN=}\NormalTok{sum)}
\NormalTok{Economic\_data}\SpecialCharTok{$}\NormalTok{ECONOMIC\_LOSS }\OtherTok{\textless{}{-}}\NormalTok{ Economic\_data}\SpecialCharTok{$}\NormalTok{PROPDMGTOTAL }\SpecialCharTok{+}\NormalTok{ Economic\_data}\SpecialCharTok{$}\NormalTok{CROPDMGTOTAL}
\NormalTok{Economic\_data }\OtherTok{\textless{}{-}}\NormalTok{ Economic\_data[}\FunctionTok{order}\NormalTok{(Economic\_data}\SpecialCharTok{$}\NormalTok{ECONOMIC\_LOSS, }\AttributeTok{decreasing =} \ConstantTok{TRUE}\NormalTok{), ]}
\NormalTok{Top10\_events\_economy }\OtherTok{\textless{}{-}}\NormalTok{ Economic\_data[}\DecValTok{1}\SpecialCharTok{:}\DecValTok{10}\NormalTok{,]}
\NormalTok{knitr}\SpecialCharTok{::}\FunctionTok{kable}\NormalTok{(Top10\_events\_economy, }\AttributeTok{format =} \StringTok{"markdown"}\NormalTok{)}
\end{Highlighting}
\end{Shaded}

\begin{longtable}[]{@{}llrrr@{}}
\toprule
& EVTYPE & PROPDMGTOTAL & CROPDMGTOTAL & ECONOMIC\_LOSS \\
\midrule
\endhead
48 & FLOOD & 143944833550 & 4974778400 & 148919611950 \\
88 & HURRICANE/TYPHOON & 69305840000 & 2607872800 & 71913712800 \\
141 & STORM SURGE & 43193536000 & 5000 & 43193541000 \\
149 & TORNADO & 24616945710 & 283425010 & 24900370720 \\
66 & HAIL & 14595143420 & 2476029450 & 17071172870 \\
46 & FLASH FLOOD & 15222203910 & 1334901700 & 16557105610 \\
86 & HURRICANE & 11812819010 & 2741410000 & 14554229010 \\
32 & DROUGHT & 1046101000 & 13367566000 & 14413667000 \\
152 & TROPICAL STORM & 7642475550 & 677711000 & 8320186550 \\
83 & HIGH WIND & 5247860360 & 633561300 & 5881421660 \\
\bottomrule
\end{longtable}

\#\#Results

\#\#\#\#Analyzing population health impact on the graph one can conclude
that TORNADOS, EXCESSIVE HEAT and FLOOD are the main contributors to
deaths and injuries out of all event types of weather events.

\begin{Shaded}
\begin{Highlighting}[]
\CommentTok{\#plotting health loss}
\FunctionTok{library}\NormalTok{(ggplot2)}
\NormalTok{g }\OtherTok{\textless{}{-}} \FunctionTok{ggplot}\NormalTok{(}\AttributeTok{data =}\NormalTok{ Top10\_events\_people, }\FunctionTok{aes}\NormalTok{(}\AttributeTok{x =} \FunctionTok{reorder}\NormalTok{(EVTYPE, PEOPLE\_LOSS), }\AttributeTok{y =}\NormalTok{ PEOPLE\_LOSS))}
\NormalTok{g }\OtherTok{\textless{}{-}}\NormalTok{ g }\SpecialCharTok{+} \FunctionTok{geom\_bar}\NormalTok{(}\AttributeTok{stat =} \StringTok{"identity"}\NormalTok{, }\AttributeTok{colour =} \StringTok{"black"}\NormalTok{)}
\NormalTok{g }\OtherTok{\textless{}{-}}\NormalTok{ g }\SpecialCharTok{+} \FunctionTok{labs}\NormalTok{(}\AttributeTok{title =} \StringTok{"Total people loss in USA by weather events in 1996{-}2011"}\NormalTok{)}
\NormalTok{g }\OtherTok{\textless{}{-}}\NormalTok{ g }\SpecialCharTok{+} \FunctionTok{theme}\NormalTok{(}\AttributeTok{plot.title =} \FunctionTok{element\_text}\NormalTok{(}\AttributeTok{hjust =} \FloatTok{0.5}\NormalTok{))}
\NormalTok{g }\OtherTok{\textless{}{-}}\NormalTok{ g }\SpecialCharTok{+} \FunctionTok{labs}\NormalTok{(}\AttributeTok{y =} \StringTok{"Number of fatalities and injuries"}\NormalTok{, }\AttributeTok{x =} \StringTok{"Event Type"}\NormalTok{)}
\NormalTok{g }\OtherTok{\textless{}{-}}\NormalTok{ g }\SpecialCharTok{+} \FunctionTok{coord\_flip}\NormalTok{()}
\FunctionTok{print}\NormalTok{(g)}
\end{Highlighting}
\end{Shaded}

\includegraphics{Weather_events_files/figure-latex/unnamed-chunk-9-1.pdf}

\#\#\#\#Analyzing economic impact on the graph one can conclude that
FLOOD, HURRICANE/TYPHOON and STORM SURGE are the main contributors to
severe economic consequences out of all event types of weather events.

\begin{Shaded}
\begin{Highlighting}[]
\CommentTok{\#plotting economic loss}
\NormalTok{g }\OtherTok{\textless{}{-}} \FunctionTok{ggplot}\NormalTok{(}\AttributeTok{data =}\NormalTok{ Top10\_events\_economy, }\FunctionTok{aes}\NormalTok{(}\AttributeTok{x =} \FunctionTok{reorder}\NormalTok{(EVTYPE, ECONOMIC\_LOSS), }\AttributeTok{y =}\NormalTok{ ECONOMIC\_LOSS))}
\NormalTok{g }\OtherTok{\textless{}{-}}\NormalTok{ g }\SpecialCharTok{+} \FunctionTok{geom\_bar}\NormalTok{(}\AttributeTok{stat =} \StringTok{"identity"}\NormalTok{, }\AttributeTok{colour =} \StringTok{"black"}\NormalTok{)}
\NormalTok{g }\OtherTok{\textless{}{-}}\NormalTok{ g }\SpecialCharTok{+} \FunctionTok{labs}\NormalTok{(}\AttributeTok{title =} \StringTok{"Total economic loss in USA by weather events in 1996{-}2011"}\NormalTok{)}
\NormalTok{g }\OtherTok{\textless{}{-}}\NormalTok{ g }\SpecialCharTok{+} \FunctionTok{theme}\NormalTok{(}\AttributeTok{plot.title =} \FunctionTok{element\_text}\NormalTok{(}\AttributeTok{hjust =} \FloatTok{0.5}\NormalTok{))}
\NormalTok{g }\OtherTok{\textless{}{-}}\NormalTok{ g }\SpecialCharTok{+} \FunctionTok{labs}\NormalTok{(}\AttributeTok{y =} \StringTok{"Size of property and crop loss"}\NormalTok{, }\AttributeTok{x =} \StringTok{"Event Type"}\NormalTok{)}
\NormalTok{g }\OtherTok{\textless{}{-}}\NormalTok{ g }\SpecialCharTok{+} \FunctionTok{coord\_flip}\NormalTok{()}
\FunctionTok{print}\NormalTok{(g)}
\end{Highlighting}
\end{Shaded}

\includegraphics{Weather_events_files/figure-latex/unnamed-chunk-10-1.pdf}

\end{document}
